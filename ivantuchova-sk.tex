%%%%%%%%%%%%%%%%%%%%%%%%%%%%%%%%%%%%%%%%%
% Friggeri Resume/CV
% XeLaTeX Template
% Version 1.0 (5/5/13)
%
% This template has been downloaded from:
% http://www.LaTeXTemplates.com
%
% Original author:
% Adrien Friggeri (adrien@friggeri.net)
% https://github.com/afriggeri/CV
%
% License:
% CC BY-NC-SA 3.0 (http://creativecommons.org/licenses/by-nc-sa/3.0/)
%
% Important notes:
% This template needs to be compiled with XeLaTeX.
%
%%%%%%%%%%%%%%%%%%%%%%%%%%%%%%%%%%%%%%%%%

\documentclass[]{friggeri-cv} % Add 'print' as an option into the square bracket to remove colors from this template for printing

\begin{document}

\header{ivana}{vantuchová}{softvérový inžinier}

%----------------------------------------------------------------------------------------
%	SIDEBAR SECTION
%----------------------------------------------------------------------------------------

\begin{aside} % In the aside, each new line forces a line break
\section{kontakt}
Přadlácká 920/24c
602 00 Brno
Česká Republika
~
(+420) 774 614 714
~
haraslinova.ivana
@gmail.com
\section{jazyky}
slovenčina | rodný
angličtina | plynulý
taliančina | plynulý
\section{progarmovanie}
PHP, C\#, CSS, HTML, JavaScript
\section{technológie}
Drupal,
REST, SOA
ASP.NET, MVC.NET
MySQL, T-SQL
\section{IDEs}
PHPStorm
Visual Studio
\section{grafika}
Adobe Illustrator
Adobe InDesign
Adobe Photoshop
Adobe Ligtroom
\end{aside}


%----------------------------------------------------------------------------------------
%	WORK EXPERIENCE SECTION
%----------------------------------------------------------------------------------------
\section{profil}

Vývojárka s magisterskou štátnicou v odbore aplikovanej informatiky a skúsenosťami s back-end vývojom. Pracovala som na vývoji REST webových služieb. Svoje teoretické vedomosti v oblasti SOA (Service-oriented architecture) som prehĺbila vďaka diplomovej práci na tému REST služby a ich verzovanie. Medzi ďalšie skúsenosti patrí vývoj webových stránok založených na platforme Drupal. Počas práce som rozvinula svoje technické znalosti Drupalu, jazyka PHP a analytické schopnosti.

\section{pracovné skúsenosti}

\begin{entrylist}
%------------------------------------------------
  \job
  {Júl 2015 \\ --- \\ doteraz}
  {Proof \& Reason s.r.o.}
  {Brno, Česká republika}
  {Digitálna agentúra zameraná na UX}
  {
    \position
    {Freelance software developer}
    {PHP, TWIG, Drupal, MySQL, Git}
    {Tvorba webových stránok na platforme Drupal so zameraním na užívateľa. Počas spolupráce s agentúrou som pracovala na vývoji webových stránok navrhnutých na mieru pre objednávateľa. Mojou úlohou je podľa dodaného prototypu a dokumentácie analyzovať požiadavky a navrhnúť riešenie pre Drupal. V ďalšej fáze som zodopvedná za implementáciu využívajúc Drupal CMS, Drupal API, PHP a TWIG.}
  }
%------------------------------------------------
\job
{September 2014 \\ --- \\ Jún 2015}
{FNZ Ltd. - Czech branch}
{Brno, Česká republika}
{Poskytovateľ investičných platforiem pre finančné inštitúcie}
{
  \position
  {Software developer}
  {C\#, VB, SOA, REST, Visual Studio, Jira}
  {Vývoj a údržba webových služieb poskytovaných interne aj tretím stranám. API bolo navrhnuté podľa architekturálneho štýlu REST. Webové služby boli implementované v jazyku C\#, vyžívali funkcionalitu jadra platformy, ktorá bola realizovaná v jazyku Visual Basic. Bola som súčasťou desaťčlenného medzinárodného týmu, úlohy boli menežované pomocou softvéru Jira. Jednalo o vývoj nových služieb, či opravy chýb v spolupráci s týmom interných testerov alebo užívateľom webovej služby.}
}
\end{entrylist}

\begin{entrylist}
\job
{August 2014}
{FNZ Ltd. - Czech branch}
{Brno, Česká republika}
{Poskytovateľ investičných platforiem pre finančné inštitúcie}
{
\position
{Letná stáž}
{C\#, Razor, JavaScript, MVC, Visual Studio}
{Vývoj internej MVC aplikácie
Vývoj systému na žiadosti a schvaľovanie dovoleniek pre zamestanancov v štvorčlennom týme. V spolupráci s vedúcim vývoja, sme analyzovali požiadavky a vytvorili návrh systému. Po schválení našeho riešenia sme pracovali na vývoji entít a kontrolerov (Model, Controller) implementovaných v jazyku C\# a pohľadov (View), kde bola použitá syntax ASPI.Net Razor s využitím JavaScriptu.}
}
%------------------------------------------------
\job
{2012 \\ --- \\ 2014}
{rôzni zákazníci}
{}
{}
{
  \position{Freelance grafický dizajnér}
  {Adobe Illustrator, Adobe InDesign, Adobe Photoshop}
  {Počas štúdií na univerzite som pracovala na niekoľkých menších grafických projektoch. Išlo najmä o návrh logotypu a vizuálnych prvkov, ale napríklad aj dizajn mobilnej aplikácie. Dlhodobejšie som spolupracovala s vydavateľstvom kníh, pre ktoré som dodávala dizajn obálky a stránok pre knihy náučnej literatúry. Moja bakalárska práca sa zaoberala návrhom vizuálnej identity pre reálneho zákazníka, výsledkom bol logotyp, propagačná grafika a grafický manuál.}
}
%------------------------------------------------
\end{entrylist}


%----------------------------------------------------------------------------------------
%	EDUCATION SECTION
%----------------------------------------------------------------------------------------

\section{vzdelanie}

\begin{entrylist}
%------------------------------------------------
\entry
{2012--2015}
{Magisterské štúdium {\normalfont Service Science, Management and Engineering}}
{Masarykova univerzita, Brno}
{\emph{Špecializácia na projektový manažment a SOA.} \\ Diplomová práca na tému SOA, verzovanie REST služieb. Analyzovala som REST služby reálnej spoločnosti a navrhla som možnosti na ich verzovanie. Hlavným cieľom bolo ponúknuť riešenie pre funkcionalitu šitú na mieru pre rôznych zákazníkov.}
%------------------------------------------------
\entry
{2009--2012}
{Bakalárske štúdium {\normalfont Aplikovaná informatika}}
{Masarykova univerzita, Brno}
{\emph{Program zaoberajúci sa informačnými technológiami, s dobrovoľnou špecializáciou na grafický dizajn.} \\ Bakalárska práca na tému vizuálnej identity pre vzdelávací projekt. Práca pokrýva celý proces tvorby logotypu a jednotnej vizuálnej identity spolu s návodom na použitie v podobe grafického manuálu.}
%------------------------------------------------
\end{entrylist}

%----------------------------------------------------------------------------------------
%	COMMUNICATION SKILLS SECTION
%----------------------------------------------------------------------------------------

%\section{communication skills}

%\begin{entrylist}
%------------------------------------------------
%\entry
%{2011}
%{Oral Presentation}
%{California Business Conference}
%{Presented the research I conducted for my Masters of Commerce degree.}
%------------------------------------------------
%\entry
%{2010}
%{Poster}
%{Annual Business Conference, Oregon}
%{As part of the course work for BUS320, I created a poster analyzing several local businesses and presented this at a conference.}
%------------------------------------------------
%\end{entrylist}

%----------------------------------------------------------------------------------------
%	INTERESTS SECTION
%----------------------------------------------------------------------------------------

\section{interests}

\textbf{professional:} webové technológie, Drupal, PHP, ASP.Net, SOA \\
\textbf{personal:} knihy, šport, jóga, cestovanie, zdravie, fotografia, dizajn

%----------------------------------------------------------------------------------------
\end{document}
